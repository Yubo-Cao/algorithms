\documentclass{standalone}
\usepackage{tikz}
\usetikzlibrary{positioning,shapes,arrows,calc}

\begin{document}

\def\sz{1cm}
% draw a hashmap stored in separate chaining
\begin{tikzpicture}
    % draw an array, from 0 to ...
    \foreach \i in {0,...,9} {
        \node[draw,rectangle,minimum width=\sz,minimum height=\sz] (array-\i) at (\i,0) {\i};
    }
    \node[draw,rectangle,minimum width=\sz,minimum height=\sz] (array-10) at (10,0) {$\ldots$};
    
    % draw a chain under array-1
    \foreach \i in {0,...,3} {
        \node[draw,circle,minimum width=\sz/2,minimum height=\sz/2] (chain-1-\i) at ($(array-1) + (0,-1.5cm -\i*\sz)$) {\i};
    }
    \draw[->] (array-1) -- (chain-1-0);
    \draw[->] (chain-1-0) -- (chain-1-1);
    \draw[->] (chain-1-1) -- (chain-1-2);
    \draw[->] (chain-1-2) -- (chain-1-3);

    % draw a chain under array 6
    \foreach \i in {0,...,2} {
        \node[draw,circle,minimum width=\sz/2,minimum height=\sz/2] (chain-6-\i) at ($(array-6) + (0,-1.5cm -\i*\sz)$) {\i};
    }
    \draw[->] (array-6) -- (chain-6-0);
    \draw[->] (chain-6-0) -- (chain-6-1);
    \draw[->] (chain-6-1) -- (chain-6-2);

    % draw a chain under array 4
    \foreach \i in {0,...,1} {
        \node[draw,circle,minimum width=\sz/2,minimum height=\sz/2] (chain-4-\i) at ($(array-4) + (0,-1.5cm -\i*\sz)$) {\i};
    }
    \draw[->] (array-4) -- (chain-4-0);
    \draw[->] (chain-4-0) -- (chain-4-1);
    
\end{tikzpicture}
\end{document}