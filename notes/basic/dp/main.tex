\documentclass{article}
\usepackage[code, outputdir=build]{common}

\title{Dynamic Programming}
\author{Yubo Cao}
\date{\today}

\begin{document}

\maketitle
\tableofcontents
\newpage

\section{Dynamic Programming}

Dynamic programming doesn't involve complicated coding techniques. It is more of mathematics and logic. The idea is to break down a problem into subproblems and solve them recursively. The key is to find the optimal substructure of the problem. The optimal substructure is the property of an optimal solution that can be broken down into optimal solutions of its subproblems. The subproblems are then solved recursively. The solutions to the subproblems are stored in a table so that they can be reused later. The solution to the original problem is then obtained by combining the solutions to the subproblems.

\subsection{Thinking Pathway}

\subsection{Bag}

Given $n$ objects with a bag that can only hold at most $w$ weight. Each of the $n$ objects has a weight $w_i$ and a value $v_i$. The goal is to fill the bag with objects so that the total value is maximized. In all those situations, one doesn't need to fill the bag to the maximum capacity. The bag can be filled partially.
\subsubsection{0-1 Bag}

Each object can only be used once.

\subsubsection{Complete Bag}

Each object can be used as many times as possible.

\subsubsection{Multiple Bag}

Each object can be used at most $m_i$ times.

\subsubsection{Grouping}

The objects are grouped into $k$ groups. Only 1 object from each group can be used.



\subsection{Linear DP}
\subsection{Interval DP}
\subsection{Counting DP}
\subsection{Memorization}

\end{document}